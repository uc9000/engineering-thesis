\documentclass[12pt,a4paper]{report}
\usepackage[utf8]{inputenc}
\usepackage{amsmath}
\usepackage{amsfonts}
\usepackage{amssymb}
\usepackage[T1]{fontenc}
\usepackage[english]{babel}
\usepackage{indentfirst}
\usepackage{hyperref}
\usepackage{minted}
\usepackage{subcaption}

\usepackage{geometry}
 \geometry{
 a4paper,
 total={160mm,247mm},
 left=25mm,
 top=25mm,
 }

\hypersetup{
    colorlinks=true,
    linkcolor=black,
    filecolor=magenta,      
    urlcolor=blue,
}

\usepackage{graphicx}
\graphicspath{ {./img/} }

\author{
	{Autor: Jordan Wiszniewski}\\
	{nr indeksu: 247859}\\
	{Promotor: Dr inż. Wojciech Myszka}\\
	{charakter pracy: projektowy}
}
\date{Rok akademicki 2021/2022}

\title{
	{\includegraphics[width = \textwidth]{logo-pwr-wm.png}}
	{\bigbreak}
	{Praca dyplomowa inżynierska}\\
	{\bigbreak}
	\textbf{Narzędzie do rysowania schematów blokowych}
}

\begin{document}

\maketitle

\tableofcontents

\chapter{Introduction}
\section{Cele pracy}
	Narzędzie powstało głównie w celach dydaktycznych -- ma ona na celu ułatwienie wizualizacji i tym samym zrozumienie początkującemu programiście działania algorytmów utworzonych w językach bazujących na języku C poprzez bezpośrednie pokazanie schematu blokowego odnoszącego się do napisanego kodu. Dodatkowo jest to przydatne narzędzie umożliwiające tworzenie schematów blokowych w łatwy i szybki sposób, wymagający jedynie podstaw programowania do dowolnego zastosowania. Narzędzie powinno również umożliwiać użytkownikom łatwe udostępnianie schematów innym użytkownikom w formie, którą łatwo można poddać dalszej edycji.
	
\section{Przykłady istniejących rozwiązaniań dla rysowania schematów blokowych typu flowchart}
	\subsection{Edytor graficzny} 
	Użycie dowolnego edytora graficznego -- jest to najmniej wydajne rozwiązanie -- rysowanie diagramów w ten sposób jest czasochłonne oraz trudne do ewentualnej edycji. Również wymaga od użytkownika dobrej znajomości edytora. Jedyną zaletą na tle innych rozwiązań jest nieograniczenie wyniku końcowego do predefiniowanych form -- daje największą swobodę.
		
	\subsection{Narzędzie do ręcznego rysowania schematów} 			
	Narzędzie specjalnie przeznaczone do rysowania za pomocą ręcznego umieszczania pojedyńczych bloczków oraz ich opisywania i łączenia np. \href{https://www.lucidchart.com/pages/examples/flowchart_software}{lucidchart} -- dużo wydajniejsze niż w przypadku użycia edytora graficznego dzięki gotowym elementom, które użytkownik może umieszczać i edytować w dość dużym zakresie.
		
	\subsection{Konwerter kodu na schemat blokowy} 	
	\begin{itemize}
	\item
	Narzędzie posiadające swoją własną składnie, stworzoną specjalnie na potrzeby rysowania diagramów (m. in. blokowych typu flowchart). Składnia sama w sobie zwykle jest dość prosta, jednak kod wymagany do stworzenia schematu blokowego w żaden sposób nie reprezentuje algorytmu, który przedstawia, a w bardziej złożonych przypadkach jest on nieczytelny co znacznie utrudnia i spowalnia pracę użytkownika. Jednym z wielu przykładów takiego narzędzia jest \href{https://mermaid-js.github.io/mermaid/#/}{Mermaid} -- używany dalej w tym projekcie jako język pośredniczący pomiędzy wprowadzanym kodem C-podobnym, a samym rysowaniem schematów na stronie HTML.
	
	\item
	Konwerter pseudo-kodu programistycznego wprost do schematu blokowego (którego wariantem jest ten projekt). Na rynku dostępna jest komercyjna wersja takiego konwertera pod nazwą \href{https://code2flow.com}{code2flow}. Zaletą tych rozwiązań jest wielokrotnie wyższa wydajność rysowania schematu oraz bezpośrednie odniesienie go do napisanego kodu co daje lepszą gwarancje poprawności rozwiązania oraz najłatwiejszą edycję spośród wszystkich dostępnych metod. Na niekorzyść tej metody działa jedynie jej mała elastyczność -- użytkownik ma niewielki wpływ na końcowy wygląd schematu, jest to w dużej mierze wstępnie zdefiniowane przez twórców tego typu narzędzi.
	\end{itemize}
	
\begin{figure}[h]
    \centering
\begin{minipage}[b]{.33\textwidth}
    \centering
   %\begin{flalign} 
   \includegraphics[width=.9\linewidth]{decyzja-while.png}
   %\end{flalign}   
\end{minipage}%
\hfill\vline\hfill
\hspace{0.02\linewidth}
\begin{minipage}[b]{0.3\textwidth}
    \centering 
%\begin{flalign}
\strut\vspace*{-\baselineskip}\newline\
C:
\begin{minted}{cpp}
  
start();
int_ i = 0;
while(i < 10){
  i++;
}
end();
  \end{minted}
%\end{flalign}
\end{minipage}%
\hfill\vline\hfill
\hspace{0.02\linewidth}
\begin{minipage}[b]{0.3\textwidth}
    \centering
    %\begin{flalign}
    \strut\vspace*{-\baselineskip}\newline\
    Mermaid:
  	\begin{minted}{python}
graph TD;
N1(["start()" ]);
N2["int i=0" ];
N5{"i<10" };
N6["i++" ];
N7(["end()" ]);
N1-->N2;
N2-->N5;
N5-->|True|N6;
N5-->|False|N7;
N6-->N5;
  	\end{minted}
   %\end{flalign}

    \end{minipage}
    \label{fig:prob3}
    \caption{Porównanie składni obu typów konwerterów, które generują ten sam schemat blokowy.}
\end{figure}
	

	
\section{Forma pracy}
	Praca ma charakter projektowy i polega na stworzeniu narzędzia, za pomocą którego będzie można w łatwy sposób narysować (wygenerować) schematy blokowe typu flowchart na podstawie elementów składniowych zaczerpniętych z języka C, takich jak:

\begin{itemize}
	\item {
		wywołanie funkcji lub przypisanie -- jako bloczek procesu
	}
	\item instrukcje warunkowe typu $if / else$ -- bloczek decyzyjny z dwiema gałęziami reprezentującymi wykonywanie kolejnych instrukcji w zależności od warunku podanego wewnątrz bloczka decyzyjnego,
	\item  instrukcja warunkowego wykonywania pętli $while$ -- jako bloczek decyzyjny, wraz z pętlą wskazującą na ten bloczek po zakończeniu instrukcji zawartych poniżej.
\end{itemize}

  Generator (podobnie jak kompilator języka C) obsługuje zagnieżdżenie w sobie powyższych instrukcji (tzn. można rozbudować drzewko decyzyjne umieszczając jedną instrukcję warunkową wewnątrz drugiej). Aplikacja jest w stanie zinterpretować jedynie wyżej wymienione instrukcje, które są wystarczające do stworzenia większości algorytmów w praktyce. Odstępstwami i jednocześnie brakami w porównaniu do języka C jest brak takich instrukcji jak: for, do-while oraz goto. Z punktu widzenia samego działania algorytmu, każdą pętle for oraz do-while można zastąpić najbardziej podstawową pętlą while -- w przypadku for: zadeklarowanie dodatkowych zmiennych przed pętlą oraz wywołanie dodatkowej instrukcji na końcu każdej iteracji pętli, natomiast w przypdaku do-while: zamiana warunku głównego pętli na wewnętrzne warunki wywołania instrukcji continue lub break. instrukcja goto została pominięta ponieważ również można ją w pełni zastąpić pętlami, używanie tej instrukcji uchodzi również za złą praktykę programistyczną.  Dodatkowo w związku z tym, że interfejsem aplikacji jest formularz HTML, wywołujący zapytania do serwera poprzez Rest API, po udostępnieniu aplikacji np. na platformie chmurowej, będzie można w łatwy sposób udostępniać utworzony schemat wraz z kodem na podstawie którego powstał za pomocą linku. Dodatkową funkcjonalnością jest również obsługa znaków Unicode, co umożliwia pisanie kodu z użyciem m.in. polskich znaków oraz zawijanie tekstu w bloczkach, zapobiegając nadmiernemu rozrostowi bloczków przy większej ilości tekstu.
  
\section{Wykorzystane narzędzia i technologie wraz z opisem}
	Wszystkie użyte technologie i narzędzia są open-source i do ich użytku (nawet komercyjnego) nie wymagają żadnych dodatkowych obligacji:

\begin{itemize}
	\item HTML5, CSS, Thymeleaf -- interfejsem aplikacji będzie prosty formularz w formie strony internetowej uzyskanej metodą szablonu HTML, komunikującej się z serwerem poprzez Rest API.	
	
	\item Mermaid -- narzędzie napisane w JavaScript umożliwiające rysowanie schematów na stronie HTML	za pomocą specjalnej składni, unikalnej dla tego narzędzia.
	
	\item Java 11 SE, Spring Boot framework -- część back-endowa aplikacji w którą m. in. wchodzą: obsługa zapytań Rest API, implementacja klas i zawartych w nich algorytmów interpretujących język C-podobny oraz zamieniających go na język wymagany przez narzędzie do rysowania schematów (Mermaid)
	
	\item Użyte narzędzia programistyczne: IntelliJ IDEA Community Edition, Visual Studio Code, system kontroli wersji Git.
		
		
\end{itemize}
	

\chapter{Types of blocks and example use cases}
\section{Process and start/end blocks}
The process block, as the name suggests, is responsible for each action. For programming code analysis, this block corresponds to a function call or an assignment operation. The start / end block in this application differs from the process block only in that it signals the starting point and all possible endings of the algorithm in a given scheme. The application automatically replaces the process blocks at the beginning and end with a start / end block. Additionally, the application ignores all $whitespace$ characters. If we want to enter a space character in the diagram block, use the character "\_" (underscore), which will be replaced. The application does not force the use of syntax elements implying that it is a function call (ending with "... ();") or an assignment (operator "="), to draw this block it is only required that only word characters  are used in any language (if having a Unicode equivalent). 
	
	\begin{figure}[H]
  \begin{subfigure}[t]{0.49\textwidth}
    \vspace{0pt}
    \includegraphics[height=6cm]{proces.png}
  \end{subfigure}\hfill
  \begin{subfigure}[t]{0.49\textwidth}
    \begin{minted}[linenos=true]{cpp}
start();
x _ = _ 2;
proces();
end();
    \end{minted}
  \end{subfigure}%
  \caption{Process, start/end blocks with the code that created those blocks }
\end{figure}

\section{Conditional block }
	\subsection{$if-else$ statement }	 
	
	  The application recognizes the syntax of the statement and creates a decision block from the given condition. Then it creates two branches responsible for actions (any number of process blocks) performed in the event of the condition being met (directly under the scope of the $if$ statement) or not being met (under the scope of the $else$ statement), and then the instructions outside the $if-else$ statement after both branches are being merged.   
		
\begin{figure}[H]
  \begin{subfigure}[t]{0.49\textwidth}
    \vspace{0pt}
    \includegraphics[height=6cm]{decyzja-if.png}
  \end{subfigure}\hfill
  \begin{subfigure}[t]{0.44\textwidth}
    \begin{minted}[xleftmargin=\parindent, linenos=true]{cpp}
start();
if(x >= 2){
  x_większe_lub_równe_2;
}
else{
  x_mniejsze_od_2;
}
end();
    \end{minted}
  \end{subfigure}%
  \caption{Conditional block created with the $if$ statement}
\end{figure}

		\subsection { $while$ statement } 
	
		Interpretacja warunku działa analogicznie jak przy instrukcji $if$, podobnie również działa dołączanie kolejnych instrukcji wykonujących się po spełnieniu warunku z tą różnicą, że ostatni proces tej gałęzi łączony jest automatycznie z bloczkiem decyzyjnym instrukcji $while$. Nie spełnienie warunku odpowiada gałęzi na której znajdują się procesy umieszczone poza scope'm  tego typu instrukcji warunkowej.
	
				\begin{figure}[H]
  \begin{subfigure}[t]{0.49\textwidth}
    \vspace{0pt}
    \includegraphics[height=6cm]{decyzja-while.png}
  \end{subfigure}\hfill
  \begin{subfigure}[t]{0.44\textwidth}
    \begin{minted}[linenos=true]{cpp}
start();
int_ i = 0;
while(i < 10){
  i++;
}
end();
    \end{minted}
  \end{subfigure}%
  \caption{Bloczek warunkowy utworzony przy użyciu instrukcji $while$}
\end{figure}	

\section{Bardziej złożony przykład}
Schemat blokowy przedstawiający (uproszczony) algorytm działania aplikacji w której został narysowany: {\smallskip}

			\begin{figure}[H]
  \begin{subfigure}{\textwidth}
    \includegraphics[width=\textwidth]{aplikacja-flowchart.png}
  \end{subfigure}\hfill
  \begin{subfigure}[t]{0.44\textwidth}
    \begin{minted}[linenos=true]{cpp}
Start;
while(Aplikacja _ uruchomiona _ na _ serwerze?){
    Wprowadzenie_ kodu_ do_ formularza;
    Przesianie_ danych_ z_ formularza_ do_ serwera_ metodą_ HTTP_ Get.;
    while(Rozpoznano_ nastepną_ instrukcję){
        Przetwórz_ następną_ instrukcję_ na_ obiekty_ pomocnicze;
        Połącz_ obiekty_ zależnie_ od_ ich_ relacji;
    }
    if(Czy_ to_ ostatnia_ instrukcja?){
    Zamiana_ obiektów_ i_ ich_ połączeń_ na_ schemat_ blokowy;
    Wyswietlenie _ schematu _ na _ stronie _ HTML;
    } else {
        Błąd _ składni;
        Wyświetlenie _ informacji _ o _ błędzie _ na _ stronie _ HTML;
    }
}
Koniec;
    \end{minted}
  \end{subfigure}%
  \caption{Przykład pokazujący obsługę zagnieżdżonych instrukcji warunkowych, zawijanie tekstu oraz wprowadzanie polskich znaków}
\end{figure}

\chapter{Description of the application source and usage}
The chapter is devoted to a general description of the operation of individual parts of the application, both from the perspective of the user and the source code.

\section{Graphical user interface}
  The graphical user interface (GUI) is a simple HTML form, available by default in a web browser window after running the application at the local host address on port 8080: 
	\smallbreak
	\href{http://localhost:8080/flowchart/}{localhost:8080/flowchart/}.
	\bigbreak	
  By default, the GUI contains a field with a generated flowchart diagram in the upper part, and a drop-down select menu, which allows the user to select the syntax of both Mermaid and C-based code, a text area for entering the code itself and a button confirming the submission of the selected syntax type and entered code via Rest API. HTTP requests of type GET handled by the Rest controller allowing two optional parameters: syntax type ($type$) -- Mermaid by default, code placed in the text field ($originalCode$) -- by default drawing two simple blocks. Any throwing of an exception by the program is handled by the controller, and then the message contained in it is forwarded to the error screen.
	
\begin{figure}[H]
  \begin{subfigure}{\textwidth}
    \centering
    \includegraphics[width=0.8\textwidth]{GUI.png}
  \end{subfigure}\hfill
  \caption{GUI displayed in web browser (Vivaldi 5)}
\end{figure}

Example flowchart available under the URL address: 
	\smallbreak
	\href{http://localhost:8080/flowchart/?type=C\&originalCode=Start();Proces();End();}{http://localhost:8080/flowchart/?type=C\&originalCode=Start();Proces();End();}
	\bigbreak
	
  An additional screen is an HTML page that displays a syntax error message and shows the code fragment where the error occurred.
	
\begin{figure}[H]
  \begin{subfigure}{\textwidth}
  \centering
    \includegraphics[width=0.8\textwidth]{error-page.png}
  \end{subfigure}\hfill
    \begin{subfigure}[t]{0.44\textwidth}
    \bigbreak
    Generated for the below code:
    \centering
    \begin{minted}[linenos=true]{cpp}    
Start();
Proces();{
End();
    \end{minted}
  \end{subfigure}%
  \caption{Example of a screen displaying syntax error}
\end{figure}


\begin{figure}[H]
	\begin{subfigure}{\textwidth}
			\begin{minted}[obeytabs=true,linenos,tabsize=1,breaklines,fontsize=\small]{java}
@GetMapping("/flowchart")
public ModelAndView flowchart(
    @RequestParam(value = "type", defaultValue = "mermaid") String type,
    @RequestParam(value = "originalCode", defaultValue = "A --> B") String code,
    ModelAndView mv
    ){
    try {
        mv.setViewName("flowchart.html");
        flowchartParser.setType(type);
        flowchartParser.code2flowchart(code);
        mv.addObject("flowchart", flowchartParser);
        return mv;
    } catch (Exception e){
        mv.setViewName("errorpage.html");
        mv.addObject("message", e.getMessage());
        return mv;
    }
}
			\end{minted}
	\end{subfigure}\hfill
  	\caption{Rest controller that supports queries with the GET method of the HTTP protocol with two parameters, which redirects to a page with a drawn diagram or an error screen}
\end{figure}

\bigbreak
To better track the course of the program and any errors in the terminal in which the application is running, detailed logs from each stage of the program's operation are displayed:
		
\begin{figure}[H]
  \begin{subfigure}{\textwidth}
  \centering
    \includegraphics[width=\textwidth]{terminal-log.png}
  \end{subfigure}\hfill
  \caption{Sample log in the terminal from the program run which ended with an error}
\end{figure}
	

\section{Lexer and parser of a C-based language}
	\subsection{Lexer} 
  The Java 11 SE language has a code lexer integrated with the parser and its main task is to recognize instructions handled by applications using regular expressions and to ensure a balanced use of both round and curly brackets in the right order, in accordance with the C programming rules. 

		\break
		
				
\begin{figure}[H]
  \begin{subfigure}{\textwidth}
			\begin{minted}[linenos=true]{java}
ELSE_PATTERN = "\\}else\\{.*";
IF_PATTERN = "if\\(.*";
WHILE_PATTERN = "while\\(.*";
EXPRESSION_PATTERN = "([\\w\\+\\-=\\(\\)\\.,<>/]+?;).*";
END_SCOPE_PATTERN = "\\}.*";
			\end{minted}
  \end{subfigure}\hfill
  \caption{List of recognizable by lexer regular expressions (REGEX patterns)}
\end{figure}
			
	\subsection{Parser} 		
  The code parser is written based on the recursive descent parser used in C/C++ compilers such as: GCC and Clang among others. It consists in recursively calling the function that reads subsequent instructions recognized by the lexer and, depending on the type of these instructions, extracts the argument content (in round brackets) and the content of the internal scope of the instruction (in curly brackets).
						
\begin{figure}[H]
  \begin{subfigure}{\textwidth}
			\begin{minted}[linenos=true]{java}

private void handleIfStatement(){
    String statement = programBuilder.toString();
    String condition = getStringInOuterParenthesis(statement);
    String expressions = getStringInOuterCurly(statement);
    programBuilder.delete(0, condition.length() + 5);
    onIfStatementEnter(condition, expressions);
    parseProgramInstruction();
}
			\end{minted}
  \end{subfigure}\hfill
  \caption{An example function that isolates its argument (condition variable) and the contents of its inner scope (expressions variable) from the $if$ statement, and then passes these values to the so-called observable method (onIfStatementEnter)}
\end{figure}
		
In this case, opening a new scope is possible only in the $if -- else$ and $while$ statements. To support the multiple nesting of those statements, an enumeration representing its statement type with each entry in its scope is pushed onto the stack and later removed from the stack with the scope exit to keep track of the current scope and assure that the correct observable methods are being self-invoked (as in the $Observer$ design pattern). In the case of instructions corresponding to process blocks (any text without the use of special characters, terminated with a semicolon), the observable method informs only about its invocation, while in the case of conditional statements ($if-else$ and $while$), there are methods dedicated to entering and exiting the area of this instruction. The default observable method only displays the basic information about the instruction in the terminal, but this information can be used freely by the version of that class with overwritten definitions of its observable methods. 
						
\begin{figure}[H]
  \begin{subfigure}{\textwidth}
		\begin{minted}[linenos=true]{java}
public void onIfStatementEnter(String condition, String expressions){
  log.info("IF entered : " + condition + " than : " + expressions);
}

public void onIfStatementExit(){
  log.info("IF exited");
}

public void onElseStatementEnter(String expressions){
  log.info("ELSE entered : " + expressions);
}
		\end{minted}
  \end{subfigure}\hfill
  \caption{An example of the default definition of observed methods for a $if/else$ statement}
\end{figure}



\section{Accelerator for building code required by Mermaid}

\begin{figure}[H]
  \begin{subfigure}{\textwidth}
  	\centering
    \includegraphics[width=\textwidth]{uml-akcelerator.png}
  \end{subfigure}\hfill
  \caption{UML diagram of classes from which the Mermaid code building accelerator was created}
\end{figure}

To facilitate the creation of flowcharts, an accelerator was created -- a set of classes that allows the programmer to easily create and manage individual elements of flowcharts, and then convert these elements in the program's memory to the target code required by Mermaid. Ultimately, it consists of a list of nodes corresponding to individual blocks and links generated on the basis of relations between these nodes. Using accessors and other public methods detailed in the UML diagram above, the programmer can easily create new nodes, assign the correct type, and connect them together.

\section{Converter from C-based language to Mermaid language}

This converter uses the previously mentioned $Observer$ design pattern. In this case, it overrides the default definitions of the observable methods previously described. Immediately after calling such an overridden method, depending on what action it was assigned to, it performs certain instructions, also using arguments passed by these methods.
	
							
\begin{figure}[H]
  \begin{subfigure}{\textwidth}
		\begin{minted}[linenos=true]{java}
@Override
public void onIfStatementEnter(String condition, String expressions){
    condition = replaceChars(condition);
    condition = wordWrap(condition, 12);
    log.info("IF entered : " + condition + " than : " + expressions);
    manager.setLastNode(fromNode);
    NodeItem scopeNode = manager
            .createDecisionNodeLinkedToLast(condition);
    scopeNodes.push(scopeNode);
    fromNode = scopeNode.getOutputs().get(0);
}

@Override
public void onIfStatementExit(){
    log.info("IF exited");
    manager.createSingleNode("#if_merge");
    manager.linkNodes(scopeNodes.peek().findLastInTree(0), 
            manager.getLastNode());
    manager.linkNodes(scopeNodes.peek().findLastInTree(1), 
            manager.getLastNode());
    scopeNodes.pop();
    fromNode = manager.getLastNode();
}

@Override
public void onElseStatementEnter(String expressions){
    log.info("ELSE entered : " + expressions);
    fromNode = scopeNodes.peek().getOutputs().get(1);
}
		\end{minted}
  \end{subfigure}\hfill
  \caption{Example of overridden observable methods that are being self-invoked as interpretation of the C-based code goes by encountering the $if-else$ statement}
\end{figure}

In the above example, after entering the $if$ (onIfStatementEnter) conditional statement, the condition is subjected to removing all special characters, replacing ``\_'' (underscore) with `` '' (white-space). Then the content of that condition is subjected to a text wrap. Later, after the initially prepared condition content, a node responsible for the decision block is created with the help of the Mermaid code building accelerator. Due to the fact that this node opens a new inner scope, it is added to the stack (scopeNodes) tracking which scope is being executed at the moment. By default, two branches come out of the decision block -- scope opening node is connected (by holding the references in the list of outputs) with two temporary helper nodes (that are removed at the end of code parsing), to the end of the first one (index = 0) the instructions contained in the inner scope are sequentially attached, read recursively according to the parser, while to the last node of the second branch (index = 1), statements are appended when the function corresponding to the $else$ statement is called. After processing all contents of the described conditional statement, another temporary helper node is created that merges both branches when leaving the inner $if-else$ scope (by calling the onIfStatementExit observable method). Finally, the decision node whose scope has just been closed is removed from the stack.

\end{document}